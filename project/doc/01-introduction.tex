\section{Introduction}
\label{ch:intro}
\noindent

As Moore's Law foretold\cite{mooreslaw}, we have been steadily increasing the ratio between computing power and metric area unit used to contain it. Computers are no longer contained to a single room, as they nowadays are able to fit in the palm of our hands. Another side effect of Moore's Law is our newfound ability to design communication protocols not only with respect to the computer, but to instead cater to the needs of developers, who might appreciate working with a protocol which is more comprehensible for humans instead of machines. While this is greatly appreciated for most use cases, there are a few outliers where these protocols are not feasible to use. One such case is the field of Internet-of-Things\cite{iotreview}. These devices are more tightly constrained in terms of resources when compared to your average personal computer, suggesting that they, depending on usage circumstances might require communication protocols designed for computers first and humans second.

\iffalse
During your previous education, you have probably come across relatively well defined problem types as formulated by teachers, textbooks and teaching aids. During project courses and exam work you are required to do a great deal of the thinking by yourself in order to define and clarify the direction of the assignment. This analysis should be presented in the report's introductory chapter. By describing the problem or problem area chosen for study and the reasons behind this choice, it should then be possible to write a general introduction to the report.
The introductory chapter relates to the content in the project plan that will be presented some weeks after the diploma work has started. The project plan should also contain a time plan for the work. The project plan can also mention some of the intended sources to be read and subsequently referred to in chapter 2, and also to contain some thoughts about the method (see chapter 3) chosen in order to approach the problem.
The introduction making up chapter 1, may also contain sub-headings underneath. Try to get to the point as soon as possible. In order to retain the reader's interest information concerning your work must be given within the first few sentences. People only requiring a quick insight into the work will often only read the report's summary, introduction and conclusions, since these sections are usually written without the inclusion of highly technical and mathematical details.
\fi

\subsection{Background and problem motivation}
\label{ch:intro:problem-motivation}

As IoT devices have been present for quite some time now, several different protocols targeting low-performance device communication have been drafted. Two of these are CoAP\cite{coap} and MQTT\cite{mqtt}. CoAP uses a REST-like model for device communication, whereas MQTT operates using a publish-subscribe model, meaning that they differ slightly in terms of what is and is not a suitable usage case for the device. Depending on the scenario, it might even be advantageous to join these two protocols when designing a larger system, depending on the specific needs of the individual components. It is not unreasonable to suggest that a system with more moving parts might very well be more complicated to author than a system with less moving parts. As such, it could be of interest to reconstruct such a situation in an attempt to later dissect it and discuss the ease of implementation for such a project, which is what this study tries to accomplish.

\iffalse
In this sub-chapter you should try to quickly engage the readers' interest in the problem area you have chosen to examine. Demonstrate that you are not only familiar with any minor technical problems, but also have an understanding of the context in which your problem emerges, that you can also describe it from a non-technical perspective, and that you are aware of the practical benefits of the technology you are examining or have knowledge of areas that your study relates to.

It is common that the first sentence contains an insightful formulation or historical retrospective. Obviously it is not possible to be absolutely certain with regards to the future, but you should express your hypothesis in a balanced and objective manner in order to appear credible.

Examples: “Humankind during historical times has… . The use of internet and cellular telephony has grown since… . The next stage in the development is expected to become… . This can lead to problems with… This study investigates if the problem can be solved with the aid of… . This technology can become especially interesting if in some years many more people…, and there is a growing demand on the market after… ”.

A technical report that is carried out on behalf of a company could start with: “Within the organization there is an increased need for… and at the same time growing problems with…. We therefore in the assignment choose to implement a preliminary study about…. A solution to this problem is urgently sought for because this can lead to a considerable reduction of costs for…, increased market shares within… and an improved work environment.”
\fi

\subsection{Overall aim}
\label{ch:intro:overall-aim}

This project aims to reconstruct a rather basic (but scaleable) scenario between several components using various different communication protocols. In total, \textbf{five} different components are present within the system, with \textbf{three} different protocols being used, depending on the context. These protocols are the two aforementioned CoAP and MQTT, as well as the WebSocket protocol.

\iffalse
(Choose one of the headline alternatives.) The project's aim is an insightful description of the direction in which you want to work, your hopes with regards to the possible outcomes of the project, and of the projects' purpose. The hypothesis does not need to be clearly defined or concrete. It can be an objective which may or may not be resolved or achieved with any degree of certainty. It can be a problem formula of a high level, which cannot be answered by the study's diagrams, tables and other objective results, but which can be discussed in the report's concluding chapter.

Examples: “the project's overall aim is to gain new knowledge within the organization about… ”. “The project's aim is to identify the general valid principles for the connection between parameter X and Y for everybody…”. “The project's aim is to find new technical solutions to problems in the following area: ….” “The project's aim is to compare technology A with technology B as a solution to the needs of C.” “The project aims to present a decision-making basis for…” “The project aims to investigate whether or not it is realistic to expect that technology A could be used for purpose B in the future.”
\fi

\subsection{Concrete and verifiable goals / Detailed problem statement}
\label{ch:intro:verifiable-goals}

The concrete and verifiable goals present for this project are as follows:

\begin{enumerate}
	\item A working system consisting of at least 5 different components (including the end client) and 3 different protocols.
	\item Partial implementations of the MQTT, CoAP and WebSocket protocol for usage within the project.
	\item Working interactions between the author's protocol implementations as well as given library counterparts.
	\item A benchmark able to assure the quality of the system.
\end{enumerate}

\iffalse
The problem- or objective statement is a verification of the proposed formula you will use to reach your objective. The questions that are specified should be answered in the report's results, and in its conclusion. The problem statement should be so clearly defined that deciding whether or not the problem has been resolved should be an easy process.

This sub-chapter is usually written after the implementation of the theoretical study in chapter 2, and should be revised at regular intervals throughout the duration of the project. The problem statement might in some cases require to be placed after the theoretical study. This way  of  writing a study may be used if it seems to be difficult for the reader to understand the concepts used. The disadvantage of such a layout is that the reader might lose interest in the subject before the core points have been stated.

Examples of problem statements useful in a scientific report are “the survey has an objective to respond to the following questions: P1: What importance has technology A compared with technology B for the performance measure Y at different values on parameter X, for cases F1 and F2? P2: Which profit gives… For mathematical definitions of X and Y, see the model in chapter 3.” It is then in chapter 3 that the objective numerical results will be specified, i.e. what will exist on the x - and y-axis in the diagram you intend to take further.

Examples of objectives for a technical report: “the survey's objective is to suggest a solution to the following technical problems: …… the survey has further objectives to verify that the solution proposal provides useable criteria and to evaluate the proposal with respect to performance measure Y.”

All technical details are reserved for the structure chapter's technical requirement specifications.
\fi

\subsection{Scope}
\label{ch:intro:scope}

Due to resource and simplicity constraints, the system will only be present on a single machine, which will detract some authenticity from the project, as a more authentic scenario would distribute the different components to different machines, depending on the use case(s) of what they are attempting to mimic.

\iffalse
Examples: “The study has its focus on…. In the survey, the effect of parameter Z is ignored, because…. The survey is distinguished by the evaluation of cases F1 and F2…. The survey's conclusions should however be generally valid for every….”
\fi

\subsection{Outline}
\label{ch:intro:outline}
Skriv det här din pajas!

Briefly describe the report's outline. “Chapter 2 describes…”

\subsection{Contributions}
\label{ch:intro:contributions}
Skriv det här din pajas!

Describe which parts of the work that you have conducted yourself, and which parts that you had help with i.e. carried out by colleagues. If the work is carried out in a group the report should then explain how the tasks were divided between authors. All co-authors should be credited in the work as a whole.
